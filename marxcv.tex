% LaTeX Template for an Academic CV by Anton Marx (anton.marx@psy.lmu.de)

% DISCLAIMER: This template is licensed under a CC-BY 4.0 license (Creative Commons https://creativecommons.org/). You may copy, distribute, and use the LaTex code, as long as you give attribution to the original authors in the code, but not in the resulting documents (e.g. PDF files).

\documentclass[letterpaper]{article}

\usepackage{hyperref}
\usepackage{geometry}

\usepackage[T1]{fontenc}

\def\name{Anton K. G. Marx} % insert name

\hypersetup{ % metadata in the PDF file properties
  colorlinks = true,
  urlcolor = black,
  pdfauthor = {\name},
  pdfkeywords = {psychology, research, education},
  pdftitle = {marx_cv},
  pdfsubject = {Curriculum Vitae},
  pdfpagemode = UseNone
}

\geometry{ % length, width and margins of document
  body={7.0in, 10.0in},
  left=0.8in,
  top=0.6in
}

\pagestyle{myheadings}
\markright{\footnotesize CV / Marx/ Updated: \today} % page header
\thispagestyle{empty}

\usepackage{sectsty}
\sectionfont{\rmfamily\mdseries\bfseries\Large} % sections font
\subsectionfont{\rmfamily\bfseries\large} % subsections font

% \ttfamily for teletype,
% \sffamily for sans serif,
% \bfseries for bold, \mdseries not bold
% \scshape for small caps,
% \normalsize, \large, \Large, \LARGE sizes.

\setlength\parindent{0em} % paragraphs not indented

\renewenvironment{itemize}{ % redefine command itemize 
  \begin{list}{}{ % with no bullets
    \setlength{\leftmargin}{2.5em} % indent
  }
}{
  \end{list}
}

\begin{document}

\begin{center} % place name at center

{\Huge \bf \name}

\end{center}

\noindent\rule{\textwidth}{0.5pt} % add line under name

\section*{CONTACT INFORMATION}

\vspace{0in}



\begin{minipage}{0.45\linewidth}
  University of Munich (LMU Munich) \\
  Department of Psychology \\
  Leopoldstraße 13 \\
  80802 Munich, Germany
\end{minipage}
\begin{minipage}{1.45\linewidth}
  \begin{tabular}{ll}
    Email: & \href{mailto:anton.marx@psy.lmu.de}{\tt anton.marx@psy.lmu.de} \\
    Phone: & 0049 (0) 89  2180 9513 \\
    Cell: &  0049 (0) 176 626 526 71 \\
    Web: & \href{https://www.psy.lmu.de/pls/people/marx/}{\tt www.psy.lmu.de/pls/people/marx} \\
  \end{tabular}
\end{minipage}



\section*{EDUCATION}

\subsection*{Since 2016 (ongoing): Ph.D., Psychology, LMU Munich}
\begin{itemize}
  \item Dissertation: {\it Advances in Research on Emotional Contagion} (working title)
    \item Supervisor: Prof. Dr. Anne C. Frenzel // Co-Supervisor: Prof. Dr. Corinna Reck
    \item Member of the Munich Center of the Learning Sciences (MCLS) Doctoral Training Program
  \end{itemize}
  
  \subsection*{Since 2015 (ongoing): Psychotherapist for Children and Adolescents}
  \begin{itemize}
    \item Post-graduate Qualification and Certification in Cognitive Behavioral Therapy (Duration: 5 years)
      \item Arbeitsgemeinschaft für Verhaltensmodifikation (AVM), Munich, Germany // Web: \href{https://https://www.avm-institute.de}{\tt www.avm-institute.de}
      \item Additional Qualifications: Trauma Therapy, Social Competence-/Self-Assertiveness-Training (SCT/SAT)
  \end{itemize}
  
  
\subsection*{2009-2015: State Exam (equivalent to M.Sc.), LMU Munich}
\begin{itemize}
  \item Subjects: School Psychology and Primary School Teaching/Educational Sciences
  \item Thesis: {\it The Relationship Between Achievement Motives and Frontal Alpha Asymmetry: An Exploratory EEG-Study} (Supervisor: PD Dr. Jörg Meinhardt)
  \item Grade/GPA: A/1.3 (with distinction) 
\end{itemize}


\section*{PROFESSIONAL EXPERIENCE}
\subsection*{Since 2016 (ongoing): Ph.D. Student and Research Associate}
\begin{itemize}
  \item LMU Munich, Department of Psychology, Psychology in the Learning Sciences // Web: \href{https://www.psy.lmu.de/pls/}{\tt www.psy.lmu.de/pls}
    \item Project leader in the DFG funded research project "Emotional Contagion in the Classroom"
    \item Principal Investigators: Prof. Dr. Frenzel, Prof. Dr. Reck, and Prof. Dr. Pekrun
    % \item Web: \href{https://www.psy.lmu.de/pls/projekt-feel/}{\tt www.psy.lmu.de/pls/projekt-feel} // DFG: \href{https://gepris.dfg.de/gepris/projekt/282833022?language=en}{\tt www.gepris.dfg.de/gepris/projekt/282833022}
    
  \end{itemize}
\subsection*{2013-2016: Student Assistant}
\begin{itemize}
  \item LMU Munich, Department of Psychology, Developmental psychology // Web: \href{https://www.psy.lmu.de/epp/}{\tt www.psy.lmu.de/epp}
    \item EEG- and Eye-Tracking-Lab (Principal Investigator: PD Dr. Jörg Meinhardt)
  \item Lab-Management and Project Coordination
\end{itemize}


\section*{PUBLICATIONS}
\subsection*{Journal Articles (Peer-Reviewed)}
\begin{itemize}
\item Müller, B. C. N.,  {\bf Marx, A. K. G. }, Paulus, M., \& Meinhardt, J. (2018). Frontal EEG Alpha Asymmetry Relates to Implicit Achievement Motives: A Pilot Study. {\it Mind, Brain and Education, 12, 2,} 82-89. doi: 10.1111/mbe.12175 
\end{itemize}

\subsection*{Conference Contributions}
\begin{itemize}
\item {\bf Marx, A. K. G.}, Frenzel, A. C., Pekrun, R., Kosovac, K., Müller, M., \& Reck, C. (2020, accepted). Examining enjoyment in the classroom using facial expression analysis. Poster submitted to the American Educational Research Association (AERA) annual conference, San Francisco, USA.
\item {\bf Marx, A. K. G.}, Frenzel, A. C., Reck, C., Kosovac, K., Müller, M., \& Pekrun, R. (2019, September). Teachers’ susceptibility to emotional contagion relates to their discrete emotions and emotional exhaustion. Paper presented at the Joint Conference of the DGPS-Sections Developmental Psychology and Educational Psychology (paEpsy), Leipzig, Germany.
\item {\bf Marx, A. K. G.}, Frenzel, A. C., Pekrun, R., Kosovac, K., Müller, M., \& Reck, C. (2019, September). Examining teachers’ and students’ enjoyment in class using automated facial emotion recognition. Paper submitted to the Joint Conference of the DGPS-Sections Developmental Psychology and Educational Psychology (paEpsy), Leipzig, Germany.
\item Kosovac, K., {\bf Marx, A. K. G.}, Frenzel., A. C., Pekrun, R., Müller, M., \& Reck, C. (2019, September). Self-reported attachment insecurity – Using the Vulnerable Attachment Style Questionnaire in adolescents. Paper presented at the Joint Conference of the DGPS-Sections Developmental Psychology and Educational Psychology (paEpsy), Leipzig, Germany.
\item {\bf Marx, A. K. G. }, Frenzel, A. C., Pekrun, R., Reck, C., \& Müller, M. (2019, August). Teachers’ and learners’ emotional experiences in class: Using automated facial action coding. Paper presented at the European Association for Research on Learning and Instruction (EARLI) annual conference, Aachen, Germany.
\item Kosovac, K., {\bf Marx, A. K. G.}, Frenzel, A. C., Pekrun, R., Müller, M., \& Reck, C. (2019, July). Attachment insecurity and emotional difficulties in adolescents. Paper presented at the International Congress of the European Society for Chilod and Adolescent Psychiatry (ESCAP), Vienna, Austria.
\item Poteat, J., {\bf Marx, A. K. G.}, \& Frenzel, A. C. (2019, June). Relationships among teachers' and students' reported and observed enjoyment in university classrooms. Paper presented at the 2nd International Conference on Children's Health, Well-being, Rights, and Education, Pristina, Kosovo.
\item {\bf Marx, A. K. G.}, Frenzel, A. C., Reck, C., Klauser, N., Müller, M., \& Pekrun, R. (2019, April). Susceptibility to emotional contagion relates to teachers' emotions and burnout. Poster presented at the American Educational Research Association (AERA) annual conference, Toronto, Canada.
\item Frenzel, A. C.*, {\bf Marx, A. K. G.*}, Pekrun, R., Reck, C., \& Müller, M. (2019, April). Teachers’ and Learners’ Emotional Experiences in Class: A Field-based Video Study. Paper presented at the American Educational Research Association (AERA) annual conference, Toronto, Canada.
\item Klauser, N., Reck, C., {\bf Marx, A. K. G.}, Müller, M., Frenzel, A. C., \& Pekrun, R. (2018, September). Bedeutung der Bindungsunsicherheit für Unterrichtsemotionen und Burnout bei Lehrkräften. Poster auf dem 51. Kongress der Deutschen Gesellschaft für Psychologie (DGPS), Frankfurt, Germany.
\item {\bf Marx, A. K. G.}, Frenzel, A. C., Klauser, N., Müller, M., Reck, C., \& Pekrun, R. (2018, September). Teachers' facial expressions of affect relate to their emotional experiences - Combining automated facial action coding with self-report. Poster auf dem 51. Kongress der Deutschen Gesellschaft für Psychologie (DGPS), Frankfurt, Germany.
\item\flushright {\it\footnotesize *these authors contributed equally to this work}
\end{itemize}


\section*{WORKSHOPS \& TALKS}
\begin{itemize}
\item Schönbrodt, F., \& {\bf Marx, A.K.G.} (2019, February). {\it Maintaining privacy with open data}. Workshop at the Open-Science-Committee (OSC), Department of Psychology, LMU Munich.
\item {\bf Marx, A.K.G.}, \& Müller, M. (2018, October). {\it Data privacy and Data security: Introduction to the EU-GDPR}. Talk at the Psychotherapeutic Outpatient Clinic for Children and Adolescents, Department of Psychology, LMU Munich.
\item {\bf Marx, A.K.G.} (2018, July). {\it Data privacy and Psychological Research}. Workshop at the Department of Psychology, Psychology in the Learning Sciences, LMU Munich.
\end{itemize}


\section*{TEACHING EXPERIENCE}
\subsection*{Study program “M.Sc. Psychology” (LMU Munich)}
\begin{itemize}
\item Emotional experiences of teachers and students in class, facial expression analysis and automated emotion recognition, research methodology and study design (quantitative research)
\end{itemize}

\subsection*{Study program “School psychology” (LMU Munich)}
\begin{itemize}
\item Counseling techniques, motivational interviewing, psychopathology and mental disorders in children and adolescents, crisis intervention techniques, cognitive behavioral therapy (basics)
\end{itemize}


\section*{PROFESSIONAL MEMBERSHIPS}
\begin{itemize}
\item[$\ast$] Deutsche Gesellschaft für Psychologie (DGPS, German Psychological Society)
\item[$\ast$] American Educational Research Association (AERA), Division C: Learning and Instruction
\item[$\ast$] European Association for Research on Learning and Instruction (EARLI), SIG Motivation \& Emotion
\end{itemize}


\section*{ADDITIONAL SKILLS}
\subsection*{Languages}
\begin{itemize}
\item German (native), English (fluent), Spanish (basics)
\end{itemize}
\subsection*{Software/Programming}
\begin{itemize}
\item R/RStudio, SPSS, Microsoft Office/Google Docs, LaTex, Markdown, HTML/CSS (basics), iMotions FACET and AFFECTIVA, Noldus Observer XT and Facereader
\end{itemize}


\section*{TRAININGS \& CERTIFICATIONS}
\subsection*{Research Practices \& Methodology}
\begin{itemize}
\item 07/2019: Advanced Power Analysis, Open-Science-Committee (OSC), LMU Munich, Germany
\item 06/2019: Research Data Management, GESIS-Institute, Cologne, Germany
\item 02/2019: Good Scientific Practices, Open-Science-Committee (OSC), LMU Munich, Germany
\item 11/2018: Data Analysis and Data Management in R, MCLS, LMU Munich, Germany
\item 07/2018: EU-GDPR: Data Privacy and Psychological Research, University of Frankfurt, Germany
\item 01/2018: Academic Writing, Graduate Center, LMU Munich, Germany
\item 02/2017: Facial Expression Analysis using the Facial-Action-Coding-System (FACS), University of Saarbrücken, Germany
\end{itemize}

\subsection*{Social Skills}
\begin{itemize}
\item 02/2015: Gender- \& Diversity-Competence, Gender Equality Office, LMU Munich, Germany
\item 2014-2015: Conflict Mediator (MuCDR),  Munich Center for Dispute Resolution, Munich, Germany
\item 08/2014: Project Management and Communication Strategies, Career Service, LMU Munich, Germany
\end{itemize}

\subsection*{Counseling \& Psychotherapeutic Interventions}
\begin{itemize}
\item 2016-2019: Video-Intervention-Therapy (VIT, George Downing), ZPP Heidelberg \& LMU Munich
\item 04/2018: Structured Clinical Interviews: Using the DIPS-Children's Version, RUB Bochum \& LMU Munich, Germany
\item 04/2016: Motivational Interviewing (MI, Miller \& Rollnick), ZPG Deggendorf, Germany
\item 2015-2016: Psychosocial Crisis Intervention \& Critical Incident Stress Management, BDP Berlin \& ASB Munich, Germany
\end{itemize}


\section*{SUPERVISION/CO-SUPERVISION}
\subsection*{Bachelor Students}
\begin{itemize}
\item 01/2019, Lara Jötten, Educational Sciences (Co-Supervisor)
\item 12/2018, Marius Wossidlo, Psychology (Supervisor)
\end{itemize}
\subsection*{Master Students (and equivalent)}
\begin{itemize}
\item 10/2019, Randy Kroker, Educational Sciences (Supervisor)
\item 09/2019, Tamara Bramböck, Psychology (Co-Supervisor)
\item 09/2019, Romy Bläse, School Psychology (Supervisor)
\item 09/2019, Marina Pfeifer, School Psychology (Supervisor)
\item 09/2019, Lysianne Simon, School Psychology (Supervisor)
\item 09/2019, Konstanze Koller, School Psychology (Supervisor)
\item 05/2019, Sarah Bramböck, School Psychology (Supervisor)
\item 04/2019, Katharina Maurer, School Psychology (Supervisor)
\item 03/2019, Laura Dietrich, School Psychology (Supervisor)
\item 12/2018, Henriette Hausdörfer, School Psychology (Supervisor)
\item 08/2018, Kaley Lesperance, Psychology (Co-Supervisor)
\item 08/2018, Raven Rinas, Psychology (Co-Supervisor)
\item 10/2018, Adrian Faruga, School Psychology (Supervisor)
\item 10/2018, Eva Staab, School Psychology (Supervisor)
\item 07/2018, Sabine Scherbauer, School Psychology (Supervisor)
\item 04/2018, Laura Kilgenstein, School Psychology (Supervisor)
\item 03/2018, Matthias Meier, School Psychology (Supervisor)
\item 03/2018, Clara Bergmann, School Psychology (Supervisor)
\item 10/2017, Anton Bach, Educational Sciences (Supervisor)
\item 09/2017, Robert Schafnitzel, School Psychology (Co-Supervisor)
\end{itemize}


\section*{REFERENCES}
\begin{itemize}
\item[$\ast$] Prof. Dr. Anne C. Frenzel, LMU Munich, Department of Psychology, \href{mailto:frenzel@psy.lmu.de}{\tt frenzel@psy.lmu.de} 
\item[$\ast$] Prof. Dr. Corinna Reck, LMU Munich, Department of Psychology, \href{mailto:corinna.reck@psy.lmu.de}{\tt corinna.reck@psy.lmu.de} 
\end{itemize}


\end{document}
